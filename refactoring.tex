\documentclass[10pt,twocolumn,letterpaper]{article}

\usepackage[scaled]{helvet}
\usepackage[T1]{fontenc}

\usepackage{amsmath}
\usepackage{amssymb}
\usepackage{cvpr}
\usepackage{epsfig}
\usepackage{tikz} 
\usepackage{graphicx}
\usepackage{stackengine}
\usepackage{times}
\usepackage{url}

%\usepackage[pagebackref=true,breaklinks=true,colorlinks,bookmarks=false,linkcolor=black,citecolor=black]{hyperref}
\usepackage{hyperref}

\cvprfinalcopy % *** Uncomment this line for the final submission

\def\cvprPaperID{2343} % *** Enter the CVPR Paper ID here
\def\httilde{\mbox{\tt\raisebox{-.5ex}{\symbol{126}}}}

\DeclareMathOperator{\argmax}{argmax}
\DeclareMathOperator{\argmin}{argmin}
\DeclareMathOperator{\far}{FAR}
\DeclareMathOperator{\val}{VAL}
\DeclareMathOperator{\fa}{FA}
\DeclareMathOperator{\ta}{TA}

\newcommand\todo[1]{\textcolor{red}{ToDo: #1}}
\newcommand\note[1]{\textcolor{red}{Note: #1}}

\newcommand\email[1]{\small{\href{mailto:#1}{\color{black}{\nolinkurl{#1}}}}}

% Pages are numbered in submission mode, and unnumbered in camera-ready
\ifcvprfinal\pagestyle{empty}\fi
\begin{document}

%%%%%%%%% TITLE
\title{Refactoring reality: Epistemic practice on abstraction}

\author{Dennis Scheiba\\
\email{dennis.scheiba@rsh-duesseldorf.de}\\
Robert Schumann Hochschule Düsseldorf - IMM - Klang und Realität\\
%Institution1 address\\
%{\tt\small firstauthor@i1.org}
% For a paper whose authors are all at the same institution,
% omit the following lines up until the closing ``}''.
% Additional authors and addresses can be added with ``\and'',
% just like the second author.
% To save space, use either the email address or home page, not both
}
\maketitle

%%%%%%%%% ABSTRACT
\begin{abstract}
\vspace{-1em}
Refactoring is a common practice in software development.
\end{abstract}

%%%%%%%%% BODY TEXT
\vspace{-1em}

\section{What is refactoring} \label{what}

Before we talk about refactoring as epistemic practice we want to take a look at some aspects of it.

There is no singleton purpose of refactoring and although its purpose is arguably existing and much important
it is difficult to give a precise definition.
The struggle between intuitively, generalization and complexity is where refactoring lives.
Attached to this is also the trade-off between opening possibilities which also always closes
other possibilities.

\subsection{Refactoring in software}

\subsection{Refactoring without Turing-completeness} \label{turing}


\section{Refactoring art}

Refactoring in a traditional sense does not change the result of the program but
allows for changes on the in- and outputs on all intermediate steps of the program.
This means that taking the naïve approach of keeping the input fixed and just run
one time the original version and the second time the refactored version shall produce
the same output.
Yet refactoring allows us to change how we approach things and therefore allows for different material
which is crucial as we could run any program in any Turing complete environment but
we still do not use Excel as a sound engine.

\subsection{Refactoring sound}

\subsection{Refactoring light}

{\small
\bibliographystyle{alpha}
\bibliography{references.bib}
}

\end{document}